%% file: template.tex = LaTeX template for article-like report 
%% init: sometime 1993
%% last: Feb  8 2015  Rob Rutten  Deil
%% site: http://www.staff.science.uu.nl/~rutte101/rrweb/rjr-edu/manuals/student-report/

%% First read ``latex-bibtex-simple-manual.txt'' at
%% http://www.staff.science.uu.nl/~rutte101/Report_recipe.html

%% Start your report production by copying this file into your XXXX.tex.
%% Small changes to the header part will make it an A&A or ApJ manuscript.

%%%%%%%%%%%%%%%%%%%%%%%%%%%%%%%%%%%%%%%%%%%%%%%%%%%%%%%%%%%%%%%%%%%%%%%%%%%%
\documentclass{aa}   %% Astronomy & Astrophysics style class
\usepackage{graphicx,natbib,url,twoopt}
\usepackage[varg]{txfonts}           %% A&A font choice
\usepackage{hyperref}                %% for pdflatex
%%\usepackage[breaklinks]{hyperref}  %% for latex+dvips
%%\usepackage{breakurl}              %% for latex+dvips
\usepackage{pdfcomment}              %% for popup acronym meanings
\usepackage{acronym}                 %% for popup acronym meanings

\hypersetup{
  colorlinks=true,   %% links colored instead of frames
  urlcolor=blue,     %% external hyperlinks
  linkcolor=red,     %% internal latex links (eg Fig)
}

\bibpunct{(}{)}{;}{a}{}{,}    %% natbib cite format used by A&A and ApJ

\pagestyle{plain}   %% undo the fancy A&A pagestyle 

%% Add commands to add a note or link to a reference
\makeatletter
\newcommand{\bibnote}[2]{\@namedef{#1note}{#2}}
\newcommand{\biblink}[2]{\@namedef{#1link}{#2}}
\makeatother

%% Commands to make citations ADS clickers and to add such also to refs
%% May 2014: they give error stops ("Illegal parameter number ..."}
%%   for plain latex with TeX Live 2013; the ad-hoc fixes added below let
%%   latex continue instead of stop within these commands.
%%   Please let me know if you know a better fix!
%%   No such problem when using pdflatex.
\makeatletter
 \newcommandtwoopt{\citeads}[3][][]{%
   \nonstopmode%              %% fix to not stop at error message in latex
   \href{http://adsabs.harvard.edu/abs/#3}%
        {\def\hyper@linkstart##1##2{}%
         \let\hyper@linkend\@empty\citealp[#1][#2]{#3}}%   %% Rutten, 2000
   \biblink{#3}{\href{http://adsabs.harvard.edu/abs/#3}{ADS}}%
   \errorstopmode}            %% fix to resume stopping at error messages 
 \newcommandtwoopt{\citepads}[3][][]{%
   \nonstopmode%              %% fix to not stop at error message in latex
   \href{http://adsabs.harvard.edu/abs/#3}%
        {\def\hyper@linkstart##1##2{}%
         \let\hyper@linkend\@empty\citep[#1][#2]{#3}}%     %% (Rutten 2000)
   \biblink{#3}{\href{http://adsabs.harvard.edu/abs/#3}{ADS}}%
   \errorstopmode}            %% fix to resume stopping at error messages
 \newcommandtwoopt{\citetads}[3][][]{%
   \nonstopmode%              %% fix to not stop at error message in latex
   \href{http://adsabs.harvard.edu/abs/#3}%
        {\def\hyper@linkstart##1##2{}%
         \let\hyper@linkend\@empty\citet[#1][#2]{#3}}%     %% Rutten (2000)
   \biblink{#3}{\href{http://adsabs.harvard.edu/abs/#3}{ADS}}%
   \errorstopmode}            %% fix to resume stopping at error messages 
 \newcommandtwoopt{\citeyearads}[3][][]{%
   \nonstopmode%              %% fix to not stop at error message in latex
   \href{http://adsabs.harvard.edu/abs/#3}%
        {\def\hyper@linkstart##1##2{}%
         \let\hyper@linkend\@empty\citeyear[#1][#2]{#3}}%  %% 2000
   \biblink{#3}{\href{http://adsabs.harvard.edu/abs/#3}{ADS}}%
   \errorstopmode}            %% fix to resume stopping at error messages 
\makeatother

%% Acronyms
\newacro{ADS}{Astrophysics Data System}
\newacro{NLTE}{non-local thermodynamic equilibrium}
\newacro{NASA}{National Aeronautics and Space Administration}

%% Add popups with meaning to acronyms 
%% NB: only show up in Adobe Reader and do not work with \input or \include
\gdef\acp#1{%
  \pdfmarkupcomment[markup=Underline,color={1 1 1},author={{#1}},opacity=0]%
  {{#1}}{{\acl{#1}}}}

%% Spectral species
\def\MgI{\ion{Mg}{I}}          %% A&A; for aastex use \def\MgI{\ion{Mg}{1}} 
\def\MgII{\ion{Mg}{II}}        %% A&A; for aastex use \def\MgII{\ion{Mg}{2}} 

%% Hyphenation
\hyphenation{Schrij-ver}       %% Dutch ij is a single character

%%%%%%%%%%%%%%%%%%%%%%%%%%%%%%%%%%%%%%%%%%%%%%%%%%%%%%%%%%%%%%%%%%%%%%%%%%%%
\begin{document}  

%% simple header.  Change into A&A or ApJ commands for those journals

\twocolumn[{%
\vspace*{4ex}
\begin{center}
  {\Large \bf FYS4150 Project 4: Numerical integration}\\[4ex]
  {\large \bf Candidate number 52 \& 55}\\[4ex]
  %{\large \bf Andreas Ellewsen$^{1}$}\\[4ex]
  %\begin{minipage}[t]{15cm}
  %      $^1$ Institute of theoretical astrophysics\\

%  {\bf Abstract.} We learned how to write nice reports \ldots 

  %\vspace*{2ex}
  %\end{minipage}
\end{center}
}]
%%%%%%%%%%%%%%%%%%%%%%%%%%%%%%%%%%%%%%%%%%%%%%%%%%%%%%%%%%%%%%%%%%%%%%%%%%%%
\section{Introduction}\label{sec:introduction}
%%%%%%%%%%%%%%%%%%%%%%%%%%%%%%%%%%%%%%%%%%%%%%%%%%%%%%%%%%%%%%%%%%%%%%%%%%%%
In this project we study the Ising model in two dimensions, without an external magnetic field. 
In its simplest form the energy is expressed  as 
\begin{equation}
 E = -J\sum_{<kl>}^N s_ks_l
\end{equation}
with $s_k = \pm 1$, $N$ is the total number of spins and $J$ is a coupling constant expressing the strength of the interaction between neighboring spins. The symbol $<kl>$ indicates that we sum over nearest neighbours only. We will assume that we have a ferromagnetic ordering, viz $j > 0$. We also assume periodic boundary conditions. This project will be solved using the Metropolis algorithm.
%%%%%%%%%%%%%%%%%%%%%%%%%%%%%%%%%%%%%%%%%%%%%%%%%%%%%%%%%%%%%%%%%%%%%%%%%%%%
\section{Analytical solution}\label{sec:analytical}
%%%%%%%%%%%%%%%%%%%%%%%%%%%%%%%%%%%%%%%%%%%%%%%%%%%%%%%%%%%%%%%%%%%%%%%%%%%%
To start with we assume that we only have two spins in each direction, such that $L = 2$, giving us a total of 4 spins.
This gives a partition function 
\begin{equation}
 Z = 12 + 4cosh(8J\beta)
\end{equation}
where $\beta = 1/kT$.
An expectation value for energy
\begin{equation}
 <E> = -\frac{32Jsinh(8J\beta)}{Z}
\end{equation}
A variance for the energy
\begin{equation}
 \sigma_E^2 = \frac{2^{10}(1+3cosh(8J\beta))}{Z^2}
\end{equation}
Expectation value for the magnetization
\begin{equation}
 <M> = 0.
\end{equation}
Expectation value for the absolute value of the magnetization
\begin{equation}
 <|M|> = \frac{8(e^{8J\beta}+2)}{Z}
\end{equation}
Variance of the magnetization
\begin{equation}
 \sigma_M^2 = \frac{2^5(e^{8J\beta}+1)}{Z} - <|M|>^2
\end{equation}
\begin{equation}
C_V = \frac{\sigma_E^2}{kT^2}
\end{equation}
and magnetic susceptibility
\begin{equation}
 \chi = \frac{\sigma_M^2}{kT}
\end{equation}

All these values will prove very useful for verifying that the program simulating larger systems is running correctly.

%%%%%%%%%%%%%%%%%%%%%%%%%%%%%%%%%%%%%%%%%%%%%%%%%%%%%%%%%%%%%%%%%%%%%%%%%%%%
\section{Simulating the L=2 system}\label{sec:simulate_analytic}
%%%%%%%%%%%%%%%%%%%%%%%%%%%%%%%%%%%%%%%%%%%%%%%%%%%%%%%%%%%%%%%%%%%%%%%%%%%%
It is now time to simulated the system using the Metropolis algorithm. The program we write computes the mean energy $<E>$, mean magnetization $<|M|>$, the specific heat $C_V$ and the susceptibility $\chi$ as functions of T. To start with we want to check that our program works correctly by comparing our results against the ones for the L=2 system. We do this for temperature $T = 1$ (in units $kT/J$).

By looking at the expectation values versus the number of Monte Carlo cycles we can estimate how many cycles are needed before we get a value close to the analytical one. Depending on what one defines as a ``good'' approximation we get either a very good match from the beginning, or a good match after about 300,000 cycles. Note that both expectation values for energy and magnetization vary only by one thousandth. See figures \ref{expecE2} and \ref{expecM2}. We include figure \ref{expecEM2} so one can see how stable the expectation values are from the beginning. Figure \ref{C_VandChiL2} confirms what we saw in the plots for magnetization and energy respectively, namely that the values stabilize after approximately 300,000 cycles.

\begin{figure}
 \includegraphics[width=.49\textwidth]{expecE2.png}
 \caption{Plot shows he computed $<E>$ versus the number of Monte Carlo cycles.In this case $T=1$ (in units of $kT/J$), and L=2}
\label{expecE2}
\end{figure}

\begin{figure}
 \includegraphics[width=.49\textwidth]{expecM2.png}
 \caption{Plot shows he computed $<|M|>$ versus the number of Monte Carlo cycles. In this case $T=1$ (in units of $kT/J$), and L=2}
\label{expecM2}
\end{figure}

\begin{figure}
 \includegraphics[width=.49\textwidth]{expecEM2.png}
 \caption{The figure shows the expectation values for energy and the absolute value of the magnetization. Note that with such a low temperature the values stabilize from the start, and the variations are too small to see at this scale.}
\label{expecEM2}
\end{figure}

Comparing the output from the program with the values we calculated earlier gives the table \ref{tab1}.
\begin{table}
 \begin{tabular}{|c|c|c|}
  \hline
  &Analytical &Numerical \\
  \hline
  $<E>$ &-1.996 & -1.996\\
  \hline
  $<|M|>$& 0.999& 0.999\\
  \hline
  $C_V$ & 0.032& 0.032\\
  \hline
  $\chi$ & 0.004& 0.004\\
  \hline
 \end{tabular}
\caption{Table of values from the analytical calculations for a system with L=2 and T = 1 in units of (kT/J). Note the precision of the numerical result.}
\label{tab1}
\end{table}

\begin{figure}
 \includegraphics[width=.49\textwidth]{C_VandChiL2.png}
 \caption{Plot of specific heat and the magnetic susceptibility as a function of Monte Carlo cycles.}
\label{C_VandChiL2}
\end{figure}


%%%%%%%%%%%%%%%%%%%%%%%%%%%%%%%%%%%%%%%%%%%%%%%%%%%%%%%%%%%%%%%%%%%%%%%%%%%%
\section{Larger systems}\label{sec:Monte Carlo import}
%%%%%%%%%%%%%%%%%%%%%%%%%%%%%%%%%%%%%%%%%%%%%%%%%%%%%%%%%%%%%%%%%%%%%%%%%%%%
Since we have now verified that our program runs correctly we want to increase the size of the lattice. We start by increasing the number of spins to $L = 20$ in each direction.
We did not study how many cycles were needed to reach the most likely state in the last section. This is something we should study so that we know how much time is needed to reach an equilibrium state, and so how long we have to wait before we start computing our expectation values. We start by making an estimate by plotting the the expectations values as function of the number of Monte Carlo cycles. This is done for $T = 1$. Figure \ref{expecEM20} show the expectation values for energy and magnetization, while figure \ref{C_VandChiL20} shows the susceptibility and the specific heat. We see that we need about 200,000 Monte Carlo cycles for the values to stabilize. In our program this has been set to 300,000 to make sure this is fulfilled.

\begin{figure}
 \includegraphics[width=.49\textwidth]{expecEM20.png}
 \caption{The figure shows the expectation values for energy and the absolute value of the magnetization for the case with $L=20$ at $T = 1$ (still in units of $kT/J$). Note that with such a low temperature the values stabilize from the start, and the variations are too small to see at this scale exactly as we saw for the $L=2$ case.}
\label{expecEM20}
\end{figure}


\begin{figure}
 \includegraphics[width=.49\textwidth]{C_VandChiL20.png}
 \caption{Plot of specific heat and the magnetic susceptibility as a function of Monte Carlo cycles. In this case $L=20$, corresponding to 400 spins. And $T = 1$. The starting state here is ordered with all spins pointing up.}
\label{C_VandChiL20}
\end{figure}

So far we have only looked at the evolution of a system that starts in an ordered state with all spins pointing up.
Next we study the same system, but this time we start in a random state. Figure \ref{disorderedL20T1} shows the evolution of this system. If one compares this to the same system with an ordered initial state (figure \ref{C_VandChiL20}) the difference is remarkable.(The expectation values stabilize too quickly to see any difference at this temperature.)

\begin{figure}
 \includegraphics[width=.49\textwidth]{disorderedL20T1.png}
 \caption{Plot of specific heat and the magnetic susceptibility as a function of Monte Carlo cycles. In this case $L=20$, corresponding to 400 spins. This time with a disordered initial state, and $T=1$.}
\label{disorderedL20T1}
\end{figure}

The next step is to increase the temperature. In figure \ref{orderedL20T24} and \ref{disorderedL20T24} you can see the magnetization and specific heat with an ordered, and a disordered initial state using $T=2.4$.

\begin{figure}
 \includegraphics[width=.49\textwidth]{orderedL20T24.png}
 \caption{Plot of specific heat and the magnetic susceptibility as a function of Monte Carlo cycles. In this case $L=20$, corresponding to 400 spins. This time with an ordered initial state, and $T=2.4$.}
\label{orderedL20T24}
\end{figure}

\begin{figure}
 \includegraphics[width=.49\textwidth]{disorderedL20T24.png}
 \caption{Plot of specific heat and the magnetic susceptibility as a function of Monte Carlo cycles. In this case $L=20$, corresponding to 400 spins. This time with a disordered initial state, and $T=2.4$.}
\label{disorderedL20T24}
\end{figure}
Note that at this temperature the figures become almost identical. There is actually a small difference in the beginning but it changes so quickly that it is nearly impossible to see with the eye.


In our algorithm there are two processes competing. There is the constant attempt to reach a lower energy state, and there is the metropolis part which accepts moves to higher energies if a random number we pick is lower than $exp(-\Delta_E\beta)$.
Because of this it would be interesting to see how many configurations that are accepted for a given temperature. This is plotted in figure \ref{acceptconfig}.
\begin{figure}
 \includegraphics[width=.49\textwidth]{acceptedconfig.png}
 \caption{The percentage of accepted configurations for a given temperature. In this case we used 1,000,000 Monte Carlo cycles for every temperature, and $L=20$.}
\label{acceptconfig}
\end{figure}
The probability P(E) of being in state E can be computed by counting the number of times a specific energy state appears in the simulation. The histograms in figure \ref{prob1} and \ref{prob2} shows the probability of each energy state for temperatures $T = 1$, and $T = 2.4$.
\begin{figure}
 \includegraphics[width=.49\textwidth]{prob_T21_MC1mill_afterthermal.png}
 \caption{The histogram shows the probability distribution of energy states for $T=2.1$.}
\label{prob1}
\end{figure}
\begin{figure}
 \includegraphics[width=.49\textwidth]{prob_T24_MC1mill_afterthermal.png}
 \caption{The percentage of accepted configurations for a given temperature.In this case we used 1,000,000 Monte Carlo cycles for every temperature, and $L=20$.}
\label{prob2}
\end{figure}
This means that if we increase the temperature the probability of reaching a high energy state gets lower and lower. Because of this there is a transition point where we go from having a lattice that is magnetized to one which is not. Note that increasing the temperature further has no effect other than minimizing the magnetization even further. 
To study this we plot the expectation values for energy and magnetization, as well as the specific heat and the susceptibility for $T=[2.1,2.7]$. The first attempt uses $L=20$. The results of this can be seen in figure \ref{expecEML20T} and \ref{C_VandChiL_20_T}. We see that the susceptibility goes up between $T=2.3$ and $T=2.4$. This could indicate a phase transition. Looking at the expectation value for the magnetization this would seem to be the case. We should do the same with a larger lattice to be certain.
\begin{figure}
 \includegraphics[width=.49\textwidth]{expecEML20T.png}
 \caption{Plot of the expectation values for magnetization and energy for the temperature range $T=[2.1,2.7]$. In this case our lattice has size $L=20$. This plot was made with a temperature step 0.01, and 1,000,000 Monte Carlo cycles for each step.}
\label{expecEML20T}
\end{figure}

\begin{figure}
 \includegraphics[width=.49\textwidth]{C_VandChi_L20_T.png}
 \caption{Plot of specific heat and the magnetic susceptibility as a function temperature. In this case $L=20$, corresponding to 400 spins.This was made with a temperature step 0.01, and 1,000,000 Monte Carlo cycles for each step.}
\label{C_VandChiL_20_T}
\end{figure}

Plotting exactly the same as before, still with 1,000,000 Monte Carlo cycles for each temperature step for $L=40$ results in figure \ref{expecEML40T} and \ref{C_VandChiL_40_T}. Here the spike in susceptibility is very clear, and the change in magnetization becomes more sudden.

\begin{figure}
 \includegraphics[width=.49\textwidth]{expecEML40T.png}
 \caption{Plot of the expectation values for magnetization and energy for the temperature range $T=[2.1,2.7]$. In this case our lattice has size $L=40$. This plot was made with a temperature step 0.01, and 1,000,000 Monte Carlo cycles for each step.}
\label{expecEML40T}
\end{figure}

\begin{figure}
 \includegraphics[width=.49\textwidth]{C_VandChi_L40_T.png}
 \caption{Plot of specific heat and the magnetic susceptibility as a function temperature. In this case $L=40$, corresponding to 400 spins.This was made with a temperature step 0.01, and 1,000,000 Monte Carlo cycles for each step.}
\label{C_VandChiL_40_T}
\end{figure}

We also include the same figures for $L=60$ (figure \ref{expecEML60T} and \ref{C_VandChiL_60_T}) and $L=80$ (figure \ref{expecEML80T} and \ref{C_VandChiL_80_T}).

\begin{figure}
 \includegraphics[width=.49\textwidth]{expecEML60T.png}
 \caption{Plot of the expectation values for magnetization and energy for the temperature range $T=[2.1,2.7]$. In this case our lattice has size $L=60$. This plot was made with a temperature step 0.01, and 1,000,000 Monte Carlo cycles for each step.}
\label{expecEML60T}
\end{figure}

\begin{figure}
 \includegraphics[width=.49\textwidth]{C_VandChi_L60_T.png}
 \caption{Plot of specific heat and the magnetic susceptibility as a function temperature. In this case $L=60$, corresponding to 400 spins.This was made with a temperature step 0.01, and 1,000,000 Monte Carlo cycles for each step.}
\label{C_VandChiL_60_T}
\end{figure}

\begin{figure}
 \includegraphics[width=.49\textwidth]{expecEML80T.png}
 \caption{Plot of the expectation values for magnetization and energy for the temperature range $T=[2.1,2.7]$. In this case our lattice has size $L=80$. This plot was made with a temperature step 0.01, and 1,000,000 Monte Carlo cycles for each step.}
\label{expecEML80T}
\end{figure}

\begin{figure}
 \includegraphics[width=.49\textwidth]{C_VandChi_L80_T.png}
 \caption{Plot of specific heat and the magnetic susceptibility as a function temperature. In this case $L=80$, corresponding to 400 spins.This was made with a temperature step 0.01, and 1,000,000 Monte Carlo cycles for each step.}
\label{C_VandChiL_80_T}
\end{figure}

The critical temperature at an infinitely large system, $L\rightarrow \infty$, can be calculated by the relation to the finite system L. 

\begin{equation}
T_C(L) - T_C(L=\infty) = aL^{-1/\nu}
\end{equation}

where a is a constant. The exact solution value of the critical temperature was obtained analytically by Lars Onsager and was found to be $T_C = 2.269$. Using the equation above, we will find our own exact solution. Since we know $T_C(L)$ for different values of L we can obtain two equations and from there find a as an expression of $T_C(L=0)$. Thus we end up with the equation 

\begin{equation}
 T_C(\infty) = \frac{T_C(L_1)L_1 - T_C(L_2)L_2}{L_1 - L_2} = 2.283
\end{equation}
where we have used that $T_C(L=20)= 2.35$ and $T_C(L=80) =2.30$. Temperature is still given in units kT/J. These temperatures are read from their respective plots.
Our own exact value of the critical temperature is fairly close to what Lars Onsager found it to be.


%%%%%%%%%%%%%%%%%%%%%%%%%%%%%%%%%%%%%%%%%%%%%%%%%%%%%%%%%%%%%%%%%%%%%%%%%%%%
\section{Conclusions} \label{sec:conclusions}
%%%%%%%%%%%%%%%%%%%%%%%%%%%%%%%%%%%%%%%%%%%%%%%%%%%%%%%%%%%%%%%%%%%%%%%%%%%%

The Ising model was motivated by the physical phenomenon ferro magnetism. In order to have ferro magnetism, atoms must have permanent magnetic moments induced by a large number of electrons which spin in the same direction. As the temperature  increases, the electron spins gets distorted and the magnetism is reduced. In contrast, decreasing the temperature will induce magnetic behavior. The electrons have stronger influence on its neighbors, and the spins align. Such transition between magnetic and non magnetic properties in a system is a phase transition. It happens at a certain temperature $T_C$. Below $T_C$ we have magnetic behavior. The Ising model allows identification of such transitions. In this project we have seen how the different properties of a system reacts as we cross the critical temperature. We have found that as we go to larger systems, the phase transition occurs at a more narrow temperature range. For example, in the $L=20$ system the transition is more smooth than for the $L=80$. Another observation is the change in susceptibility which blows up as the system gets bigger. Close to the critical temperature, $\chi$ diverges and peaks at $T = T_C$. The susceptibility can be interpreted to be the quantitative measure of the extent of magnetization. 
Thus the larger the system, the larger the magnetic field strength.

%%%%%%%%%%%%%%%%%%%%%%%%%%%%%%%%%%%%%%%%%%%%%%%%%%%%%%%%%%%%%%%%%%%%%%%%%%%%
\section{Project files}\label{sec:files}
%%%%%%%%%%%%%%%%%%%%%%%%%%%%%%%%%%%%%%%%%%%%%%%%%%%%%%%%%%%%%%%%%%%%%%%%%%%%
The project files include a readme file detailing how to use the c++ program correctly. 
Since the program is used for many different things we have implemented a parameter for choosing which method one wants to use when running the program. The user can also choose whether he or she wants the system to start in an ordered or disordered state by adjusting a parameter. 

There is also included in the project files 4 different plotting files, each corresponding to the method used for creating the output file. 

To make sure the reader gets the same output as the authors when running the program, a benchmark folder containing output files created with set parameters is included. The reader is encouraged to use the files provided rather than make their own as some of the files take a tremendous amount of time to create unless the reader has access to a lot of computing power. This is especially true for method 2 when working with large systems and many temperatures.

%%%%%%%%%%%%%%%%%%%%%%%%%%%%%%%%%%%%%%%%%%%%%%%%%%%%%%%%%%%%%%%%%%%%%%%%%%%%
%\begin{acknowledgements}
%\end{acknowledgements}

%%%%%%%%%%%%%%%%%%%%%%%%%%%%%%%%%%%%%%%%%%%%%%%%%%%%%%%%%%%%%%%%%%%%%%%%%%%%
%% references
\section{References}
Computational Physics, Lecture notes Fall 2015, Morten Hjorth-Jensen

%\bibliographystyle{aa-note} %% aa.bst but adding links and notes to references
%\raggedright              %% only for adsaa with dvips, not for pdflatex
%\bibliography{XXX}          %% XXX.bib = your Bibtex entries copied from ADS

\end{document}