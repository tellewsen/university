% !TeX spellcheck = en_US
\documentclass[a4paper,12pt]{article}
\usepackage[utf8x]{inputenc}
\usepackage{wrapfig}
\usepackage{graphicx}
\usepackage{float}
\usepackage{listings}
\usepackage{amsmath}
\usepackage{caption}
\usepackage{subcaption}
\usepackage[usenames,dvipsnames,svgnames,table]{xcolor}
\usepackage{datetime}
\usepackage{fancyhdr}
\usepackage{tikz}
\pagestyle{fancy}
\renewcommand{\thesubsection}{\thesection.\alph{subsection}}

% Title Page
\title{Home exam FYS3140}
\author{Candidate 11}

\fancyhead[L]{Candidate 11}
\fancyhead[C]{Home exam}
\fancyhead[R]{Spring 2015}

\begin{document}

\maketitle
\tableofcontents
\newpage

\section{Problem 1: Fourier transforms and special functions}
Some special functions are defined in terms of infinite sums. For example, the Riemann zeta function,
\begin{equation}
\zeta(s) = \sum_{n=1}^{\infty}\frac{1}{n^s}, (\Re(s) > 1)
\end{equation}
and similarly the Dirichlet eta function
\begin{equation}
\zeta(s) = \sum_{n=1}^{\infty}(-1)^{n-1}\frac{1}{n^s}, (\Re(s) > 0).
\end{equation}
One application of Fourier series is the possibility to compute certain values of this kind of functions.\\
\subsection{}
Compute the Fourier series of $f(x) = x^2$ with basic interval $ -\pi < x < \pi$.\\

\textbf{Solution:}\\
Since $f(x)$ is an even function I start by setting $b_n = 0$ for all $n$.
The Fourier series can then be written
\begin{equation}
f(x) = \frac{a_0}{2} + \sum_{n=1}^{\infty}a_n cos(nx)
\end{equation}
and the coefficients can be found with 
\begin{equation}
a_n = \frac{2}{\pi}\int_{0}^{\pi}x^2cos(nx)dx.
\end{equation}
First I find $a_0$:

\begin{equation*}
\begin{aligned}[\left]
a_0 &= \frac{2}{\pi}\int_{0}^{\pi}x^2dx\\
a_0 &= \frac{2}{\pi}\bigg[\frac{x^3}{3}\bigg]_{0}^{\pi}\\
a_0 &= \frac{2}{\pi}\frac{\pi^3}{3}\\
a_0 &= \frac{2\pi^2}{3}
\end{aligned}
\end{equation*}
The rest of the coefficients can be found by using integration by parts on the integral in equation (4), and setting $u=x^2$, $u' = 2x$, $v' = cos(nx)$, $v = \frac{1}{n}sin(nx)$:

\begin{equation*}
\int x^2cos(nx)dx = \frac{x^2}{n}sin(nx)  - \frac{2}{n}\int xsin(nx)dx\\
\end{equation*}

Where the integral on the right hand side can be solved with integration by parts. Setting $u=x$, $u' = 1$, $v' = sin(nx)$, $v = \frac{-1}{n}cos(nx)$ gives:

\begin{equation*}
\begin{aligned}[\left]
\int xsin(nx)dx &= -\frac{x}{n}cos(nx) +\frac{1}{n}\int cos(nx)dx\\
&=-\frac{x}{n}cos(nx) + \frac{1}{n^2}sin(nx)
\end{aligned}
\end{equation*}

Inserting this back into (4) yields
\begin{equation*}
\begin{aligned}[\left]
a_n &= \frac{2}{\pi}\bigg[\frac{x^2}{n}sin(nx) - \frac{2}{n}\bigg(\frac{-x}{n}cos(nx)+\frac{1}{n^2}sin(nx)\bigg) \bigg]_{0}^{\pi}\\
    &= \frac{2}{\pi}\bigg(\frac{2\pi}{n^2}cos(n\pi)\bigg)\\
    &= \frac{4}{n^2}cos(n\pi)\\
    &= \frac{4(-1)^n}{n^2}
\end{aligned}
\end{equation*}
and thus we have found that
\begin{equation}
f(x) = x^2 = \frac{\pi^2}{3} + \sum_{n=1}^{\infty}\frac{4(-1)^n}{n^2}cos(nx)
\end{equation}

\subsection{}
Use your result to show that 
\begin{equation}
\zeta(2) = \frac{\pi^2}{6}
\end{equation}
and
\begin{equation}
\eta(2) = \frac{\pi^2}{12}\\
\end{equation}

\textbf{Solution:}\\
We start with $\zeta(2)$. Using the definition of the function:
$$\zeta(2) = \sum_{n=1}^{\infty}\frac{1}{n^2}$$

If one now takes equation (6) and sets $x = \pi$ one finds that

\begin{equation*}
\begin{aligned}[\left]
\pi^2 &= \frac{\pi^2}{3} + \sum_{n=1}^{\infty}\frac{4(-1)^n}{n^2}cos(n\pi)\\
\pi^2 &= \frac{\pi^2}{3} + \sum_{n=1}^{\infty}\frac{4(-1)^n(-1)^n}{n^2}\\
\pi^2 &= \frac{\pi^2}{3} + \sum_{n=1}^{\infty}\frac{4}{n^2}\\
\pi^2 - \frac{\pi^2}{3} &=  4\sum_{n=1}^{\infty}\frac{1}{n^2}\\
\sum_{n=1}^{\infty}\frac{1}{n^2} &=  \frac{3\pi^2 - \pi^2}{12}\\
\sum_{n=1}^{\infty}\frac{1}{n^2} &=  \frac{\pi^2}{6}\\
\zeta(2) &= \frac{\pi^2}{6}
\end{aligned}
\end{equation*}
Which was what I wanted to show.\\
From the definition of $\eta$ we have
$$\eta(2) = \sum_{n=1}^{\infty}\frac{(-1)^{n-1}}{n^2}$$
Again I use equation (6) but this time I set $x = 0$ and get the following:
\begin{equation*}
\begin{aligned}[\left]
0 &= \frac{\pi^2}{3} + \sum_{n=1}^{\infty}\frac{4(-1)^n}{n^2}\\
-\frac{\pi^2}{3} &=  4\sum_{n=1}^{\infty}\frac{(-1)^n}{n^2}\\
-\frac{\pi^2}{3} &=  -4\sum_{n=1}^{\infty}\frac{(-1)^{n-1}}{n^2}\\
\sum_{n=1}^{\infty}\frac{(-1)^{n-1}}{n^2} &=  \frac{\pi^2}{12}\\
\eta(2) &= \frac{\pi^2}{12}
\end{aligned}
\end{equation*}
Which was what I wanted to show.

\section{Problem 2: Contour integral}
In this problem we will explore the complex integral
\begin{equation}
I = \oint_\gamma \frac{dz}{1+z^n}; n \geq 2 \\
\end{equation}
along the closed contour $\Gamma = \gamma_1 + \gamma_2 + \gamma_3$ where\\
$\gamma_1$ : Straight line along the x-axis from the origin to $x = R$\\
$\gamma_2$ : Arc of a circle of radius R from $z = x = R$ to $z = R e^{2\pi i/n}$\\
$\gamma_3$ : Straight line from $z = Re^{2\pi i/n}$ to the origin,\\
where we will let $R \rightarrow \infty$ eventually.
\subsection{}
Sketch the integration contour.

\textbf{Solution:}\\
From $\gamma_2$ it is clear that one starts with a half circle, and that this circle is made smaller by increasing $n$. Since it is clear that the integration contour is dependent on $n$ I've chosen to make a figure for $n=4$. 
\begin{figure}[H]
	\begin{center}
	\begin{tikzpicture}
	%Draw axis
	\draw[->] (-2.5,2) -- (6.5,2) node[anchor=north west] {\text{$\Re$}};
	\draw[->] (2,-2.5) -- (2,6.5) node[anchor=south east] {\text{$\Im$}};
	%Draw integration countour
	\draw [red,thick,->](2,2) -- (5,2) node[anchor=north] {\text{$\gamma_1$}};
	\draw [red,thick,->](5,2) arc (0:90:3cm) node[anchor=south west] {\text{$\gamma_2$}};
	\draw [red,thick,->](2,5) -- (2,2) node[anchor=south east] {\text{$\gamma_3$}};
	%Mark singularities
	\node (a) at (3,3) {};
	\fill [red] (a) circle (2pt);
	\node (b) at (1,3) {};
	\fill [red] (b) circle (2pt);
	\node (c) at (1,1) {};
	\fill [red] (c) circle (2pt);
	\node (d) at (3,1) {};
	\fill [red] (d) circle (2pt);
	\end{tikzpicture}
	\caption{The integration path when $n=4$.}
	\end{center}
\end{figure}

\subsection{}
Find all singularities of the integrand and indicate them in your figure from a).

\textbf{Solution:}\\
Since the integrand is $\frac{dz}{1+z^n}$, one gets a singularity when 
\begin{equation*}
\begin{aligned}[\left]
z^n &= -1 = e^{i\pi(1+2k)}\\
z   &= e^{\frac{i\pi(1+2k)}{n}}; k = 0,1,...,n-1\\
\end{aligned}
\end{equation*}
From this it is clear that the number of, and the position of, the singularities is dependent on $n$.
As an example I get for $n=2$:
\begin{equation*}
\begin{aligned}[\left]
z_1 &= e^{\frac{i\pi}{2}} = i\\
z_2 &= e^{\frac{3i\pi}{2}} = -i
\end{aligned}
\end{equation*}
while for $n=3$ I get:
\begin{equation*}
\begin{aligned}[\left]
z_1 &= e^{\frac{i\pi}{3}} \\
z_2 &= e^{\frac{3i\pi}{3}} = -1\\
z_3 &= e^{\frac{5i\pi}{3}}
\end{aligned}
\end{equation*}

In fact, there are $n$ singularities, and if one makes a straight line from one to the next they will form an n-sided polygon. Since it would be a complete mess to mark of singularities for all $n$, I've chosen to mark the ones for $n=4$, since that is the $n$ I chose for the figure earlier.

\subsection{}
Compute the integral I in the limit $R \rightarrow \infty$.

\textbf{Solution:}\\
From the residue theorem we have that
\begin{equation}
 \int_\Gamma f(z)dz = 2\pi i\sum_N Res[f(z),z_n].
\end{equation}
So instead of calculating the integral I can just find all the residues inside $\Gamma$. Now I need to find how many are inside $\Gamma$ depending on what $n$ is.
For a singularity to be inside $\Gamma$ the following must be true:
\begin{equation*}
 \begin{aligned}[\left]
  \frac{2\pi}{n} &\ge \frac{\pi(1+2k)}{n}\\
  2\pi &\ge \pi(1+2k)\\
  2 &\ge 1+2k\\
  \frac{1}{2} &\ge k\\
  k &\le \frac{1}{2} 
 \end{aligned}
\end{equation*}
and since k increases in integer steps, this only happens when $k = 0$.
This means that 
$$\sum_N Res[f(z),z_n] = Res[f(z),z_0].$$
And this means that I can conclude that no matter what $n$ is, there will always be a singularity inside the contour $\Gamma$ at $z_0 = e^{i\pi/n}$.
To find this residue I use that for rational functions $f(z) = \frac{P(z)}{Q(z)}$, we have 
\begin{equation}
Res[f(z),z_0] = \frac{P(z_0)}{Q'(z_0)}. 
\end{equation}
In our case $P(z) = 1$, and $Q(z) = 1+z^n$. Thus $Q'(z) = nz^{n-1}$, and 
\begin{equation*}
\begin{aligned}[\left]
I &= 2\pi i Res[f(z),z_0] = 2\pi i\frac{1}{nz_0^{n-1}} = \frac{2\pi i}{n}e^{-\frac{i\pi(n-1)}{n}}\\
I &= -\frac{2\pi i}{n}e^{\frac{i\pi}{n}}
\end{aligned}
\end{equation*}

\subsection{}
Examine the contribution to the integral from the circle arc $\gamma_2$ as $R \rightarrow \infty$. (Hint: perform an upper bound estimate)

\textbf{Solution:}\\
I want to estimate an upper bound for $\int_{\gamma_2} \frac{1}{1+z^n}dz$.
This can be calculated by 
\begin{equation}
 \bigg|\int_{\gamma_2}f(z)dz\bigg| \le ML
\end{equation}
Where $L$ is the length of the contour, and $M$ is the maximum value of $|f(z)|$ on the contour $\Gamma$. 

$L$ is straightforward to find:
\begin{equation*}
\begin{aligned} [\left]
dL &= Rd\theta\\
 L &= \frac{R2\pi}{n}
\end{aligned}
\end{equation*}

Finding M:
\begin{equation*}
\begin{aligned}[\left]
\bigg|\frac{1}{1+z^n}\bigg| &= \frac{|1|}{|1+z^n|} \le \frac{1}{|z^n|-1} = \frac{1}{R^n-1} = M\\
M &= \frac{1}{R^n-1}
\end{aligned}
\end{equation*}
And thus
\begin{equation*}
\begin{aligned}[\left]
 \bigg|\int_{\gamma_2}f(z)dz\bigg| \le& \lim_{R\rightarrow\infty}\frac{R2\pi}{n}\frac{1}{R^n-1}\\
\le& \lim_{R\rightarrow\infty}\frac{2\pi}{n^2R^{n-1}}\\
\le& 0\\
\bigg|\int_{\gamma_2}f(z)dz\bigg| &\le 0\\
\int_{\gamma_2}f(z)dz &= 0.\\
\end{aligned}
\end{equation*}
Where I've used L'Hopital's rule to calculate the limit. So the conclusion is that $\gamma_2$ contributes nothing to the integral as long as $R \rightarrow \infty$.
\subsection{}
Using your results from c) and d), show that the real integral
\begin{equation}
 \int_0^\infty \frac{dx}{1+x^n} = \frac{\pi/n}{sin(\pi/n)}
\end{equation}
(Hint: Find a suitable parametrization along $\gamma_3$).

\textbf{Solution:}\\
I found earlier that
$$I = \frac{-2\pi i}{n}e^{i\pi/n}$$
but one can also write the integral for I as
\begin{equation*}
 \begin{aligned}[\left]
  I =& \int_{\gamma_1}\frac{dx}{1+x^n} + \int_{\gamma_2} \frac{dz}{1+z^n} + \int_{\gamma_3}\frac{dz}{1+z^n}\\
   =& \lim_{R\rightarrow\infty} \int_0^R\frac{dx}{1+x^n} + 0 + \lim_{R\rightarrow\infty} e^{2i\pi/n}\int_R^0\frac{dx}{1+x^n}\\
   =& \int_0^\infty\frac{dx}{1+x^n} - e^{2i\pi/n}\int_0^\infty\frac{dx}{1+x^n}\\
   =& (1-e^{2i\pi/n})\int_0^\infty\frac{dx}{1+x^n}.
 \end{aligned}
\end{equation*}
Since I actually want to find the integral in this last expression, I set this expression equal to the one I found earlier for $I$:
\begin{equation*}
 \begin{aligned}[\left]
  \frac{-2\pi i}{n}e^{i\pi/n}   =& (1-e^{2i\pi/n})\int_0^\infty\frac{dx}{1+x^n}\\
  \int_0^\infty\frac{dx}{1+x^n} =& \frac{\frac{-2\pi i}{n}e^{i\pi/n}}{1-e^{2i\pi/n}}\\
=& \frac{\pi}{n} \frac{2i}{{e^{\frac{i\pi}{n}}-e^\frac{-i\pi}{n}}}\\
\int_0^\infty\frac{dx}{1+x^n}=& \frac{\pi/n}{sin(\pi/n)}.
\end{aligned}
\end{equation*}
Which was what I wanted to show.

\section{Problem 3: Fröbenius method}
Here you will use the Fröbenius method to address the differential equation
\begin{equation}
 x^2y'' +xy' + (x^2-\frac{1}{4})y = 0
\end{equation}

\subsection{}
Set up and solve the indicial equation

\textbf{Solution:}\\
Start by assuming a solution of the form
$$y(x) = \sum_{n=0}^\infty a_nx^{n+s}$$
which has first derivative
$$y'(x) = \sum_{n=0}^\infty (n+s)a_nx^{n+s-1}$$
and second derivative
$$y''(x) = \sum_{n=0}^\infty (n+s)(n+s-1)a_nx^{n+s-2}.$$
Inserting these into the DE (14):
\begin{equation*}
\sum_{n=0}^\infty (n+s)(n+s-1)a_nx^{n+s} + \sum_{n=0}^\infty (n+s)a_nx^{n+s} + (x^2-\frac{1}{4})\sum_{n=0}^\infty a_nx^{n+s} = 0.
\end{equation*}
Since this has to hold for every power of $x$ we start with the first one, $x^s:$
\begin{equation*}
\begin{aligned}[\left]
s(s-1)a_0 + sa_0 + 0 -\frac{1}{4}a_0 &= 0 ; a_0 \neq 0 \\
s^2 -s +s - \frac{1}{4}  &= 0 \\
s^2 &= \frac{1}{4}\\
s  &= \pm\frac{1}{2} 
\end{aligned}
\end{equation*}

\subsection{}
Find the general solution for $y(x)$.

\textbf{Solution:}\\
Since $s1-s2 = integer$, there is a possibility that the smallest of them will generate a complete solution.
Because of this I want to try using $s = -1/2$, and see if I can find the coefficients $a_n$. 
To do this I start with the lowest power of x, e.g $x^{-1/2}$:
\begin{equation*}
\begin{aligned}[\left]
\bigg(-\frac{1}{2}\bigg)\bigg(-\frac{1}{2}-1\bigg)a_0 - \frac{1}{2}a_0 - \frac{1}{4}a_0 &= 0\\
\bigg(\frac{3}{4} - \frac{3}{4}\bigg)a_0 &= 0
\end{aligned}
\end{equation*}
Which means $a_0$ is undetermined.

Next is $x^{1/2}$:
\begin{equation*}
\begin{aligned}[\left]
\bigg(1-\frac{1}{2} \bigg)\bigg(1-\frac{1}{2}-1\bigg)a_1 + \bigg(1-\frac{1}{2}\bigg)a_1 -\frac{1}{4}a_1 &= 0\\
-\frac{1}{4}a_1 +\frac{1}{2}a_1 -\frac{1}{4}a_1 &= 0  \\
\bigg(\frac{1}{2}-\frac{1}{2} \bigg)a_1 &= 0
\end{aligned}
\end{equation*}
Which means $a_1$ is also undetermined. And since I now have two undetermined coefficients, this has produced the complete solution.

Next thing then is to find an equation for $a_n$ with $n \ge 2$:
\begin{equation*}
\begin{aligned}[\left]
\bigg(n-\frac{1}{2}\bigg)\bigg(n-\frac{1}{2}-1\bigg)a_n + \bigg(n-\frac{1}{2}\bigg)a_n +a_{n-2} -\frac{1}{4}a_n &= 0\\
\bigg(n-\frac{1}{2}\bigg)\bigg(n-\frac{3}{2}\bigg)a_n + \bigg(n-\frac{1}{2}\bigg)a_n -\frac{1}{4}a_n &= -a_{n-2}\\
\bigg(n^2 -\frac{3n}{2} -\frac{n}{2} +\frac{3}{4} + n - \frac{1}{2} - \frac{1}{4}\bigg)a_n &= -a_{n-2}\\
\bigg(n^2 - n\bigg)a_n &= -a_{n-2}\\
a_n &= -\frac{a_{n-2}}{n(n - 1)}
\end{aligned}
\end{equation*}

This means that the complete solution is
\begin{equation}
y(x) = a_0x^{-1/2} +a_1x^{1/2} - \sum_{n=2}^{\infty}\frac{a_{n-2}}{n(n-1)}x^{n-1/2}
\end{equation}
or if one doesn't like infinite sums one can write out the start of the sum
\begin{equation*}
\begin{aligned}[\left]
y(x) =& a_0x^{-1/2} +a_1x^{1/2}-\frac{a_0}{2!}x^{3/2}-\frac{a_1}{3!}x^{5/2} +\frac{a_0}{4!}x^{7/2} + \frac{a_1}{5!}x^{9/2} + ...\\
\end{aligned}
\end{equation*}
and notice that this can be written
\begin{equation}
\begin{aligned}[\left]
y(x) =& x^{-1/2}[a_0 cos(x) + a_1 sin(x)]
\end{aligned}
\end{equation}

\subsection{}
Show that the solution can be written as
\begin{equation}
 y(x) = AJ_{1/2}(x) + BJ_{-1/2}(x) 
\end{equation}
where A and B are constants, and $J_{\pm1/2}(x)$ are Bessel functions of the first kind, of order $\pm\frac{1}{2}$. Bessel functions are important in a number of physics applications, and information about them can be found in mathematical tables.

\textbf{Solution:}\\
To start with I need to know what these Bessel functions look like. It turns out that Bessel functions of the first kind of order p are defined as 
\begin{equation}
J_p(x) = \sum_{n=0}^{\infty}\frac{(-1)^n}{\Gamma(n+1)\Gamma(n+1+p)}\bigg(\frac{x}{2}\bigg)^{2n+p}
\end{equation}
which means that
\begin{equation}
J_{1/2}(x) = \sum_{n=0}^{\infty}\frac{(-1)^n}{\Gamma(n+1)\Gamma(n+3/2)}\bigg(\frac{x}{2}\bigg)^{2n+1/2}
\end{equation}
and 
\begin{equation}
J_{-1/2}(x) = \sum_{n=0}^{\infty}\frac{(-1)^n}{\Gamma(n+1)\Gamma(n+1/2)}\bigg(\frac{x}{2}\bigg)^{2n-1/2.}
\end{equation}
I will also need that 
\begin{equation}
cos(x) = \sum_{n=0}^{\infty}(-1)^n\frac{x^{2n}}{(2n)!}
\end{equation}
and 
\begin{equation}
sin(x)= \sum_{n=0}^{\infty}(-1)^n\frac{x^{2n+1}}{(2n+1)!}.
\end{equation}
First I write $y(x)$ with these other forms for $sin(x)$ and $cos(x)$:

\begin{equation*}
\begin{aligned}[\left]
y(x) &= x^{-1/2}\bigg[a_0\sum_{n=0}^{\infty}(-1)^n\frac{x^{2n}}{(2n)!} + a_1\sum_{n=0}^{\infty}(-1)^n\frac{x^{2n+1}}{(2n+1)!}\bigg]\\
y(x) &= a_0\sum_{n=0}^{\infty}(-1)^n\frac{x^{2n-1/2}}{(2n)!} + a_1\sum_{n=0}^{\infty}(-1)^n\frac{x^{2n+1/2}}{(2n+1)!}\\
y(x) &= a_0\sum_{n=0}^{\infty}\frac{2^{2n-1/2}(-1)^n}{(2n)!}\bigg(\frac{x}{2}\bigg)^{2n-1/2} + a_1\sum_{n=0}^{\infty}\frac{2^{2n+1/2}(-1)^n}{(2n+1)!}\bigg(\frac{x}{2}\bigg)^{2n+1/2}
\end{aligned}
\end{equation*}
For this to be the same as equation (16), the following must be true:
\begin{equation}
\frac{2^{2n-1/2}a_0}{(2n)!} = \frac{B}{\Gamma(n+1)\Gamma(n+1/2)}
\end{equation}
and
\begin{equation}
\frac{2^{2n+1/2}a_1}{(2n+1)!} = \frac{A}{\Gamma(n+1)\Gamma(n+3/2)}
\end{equation}
Start with (22):
\begin{equation*}
\begin{aligned}[\left]
&\frac{2^{2n-1/2}a_0}{(2n)!} = 2^{-1/2}a_0 \frac{2^{2n}}{(2n)!}\\
&= 2^{-1/2}a_0 \frac{2^{2n}}{(2n)(2n-1)(2n-2)(2n-3)...3\times 2\times 1}\\
&= 2^{-1/2}a_0 \frac{1}{(n)(n-1/2)(n-1)(n-3/2)...3/2\times 1\times 1/2}\\
&= 2^{-1/2}a_0 \frac{\Gamma(n)\Gamma(n-\frac{1}{2})\Gamma(n-1)...\Gamma(1)\Gamma(\frac{1}{2})} {(n)\Gamma(n)(n-\frac{1}{2})\Gamma(n-\frac{1}{2})(n-1)\Gamma(n-1)...\Gamma(1)\frac{1}{2}\Gamma(\frac{1}{2})}\\
&= 2^{-1/2}a_0 \frac{\Gamma(n)\Gamma(n-\frac{1}{2})\Gamma(n-1)...\Gamma(1)\Gamma(\frac{1}{2})} {\Gamma(n+1)\Gamma(n+\frac{1}{2})\Gamma(n)...\Gamma(1)1/2\Gamma(\frac{1}{2})}\\
&= 2^{-1/2}a_0 \frac{\Gamma(1)\Gamma(\frac{1}{2})} {\Gamma(n+1)\Gamma(n+\frac{1}{2})}\\
&= 2^{-1/2}a_0 \frac{\pi^{1/2}} {\Gamma(n+1)\Gamma(n+\frac{1}{2})}\\
\end{aligned}
\end{equation*}
which works if I define 
\begin{equation}
B = a_0\bigg(\frac{\pi}{2}\bigg)^{1/2}.
\end{equation}
This is okay, since all I've actually done to get $B$ is multiply $a_0$ by some constant.
Similarly for (23):
\begin{equation*}
\begin{aligned}[\left]
&\frac{2^{2n+1/2}a_1}{(2n+1)!} = 2^{1/2}a_1 \frac{2^{2n}}{(2n)!}\\
&= 2^{1/2}a_1 \frac{2^{2n}}{(2n+1)(2n)(2n-1)(2n-2)(2n-3)...3\times 2\times 1}\\
&= 2^{1/2}a_1 \frac{1}{(n+1/2)(n)(n-1/2)(n-1)(n-3/2)...3/2\times 1\times 1}\\
&= 2^{1/2}a_1 \frac{\Gamma(n+\frac{1}{2})\Gamma(n)\Gamma(n-\frac{1}{2})...\Gamma(1)} {(n+1/2)\Gamma(n+\frac{1}{2})(n)\Gamma(n)(n-\frac{1}{2})\Gamma(n-\frac{1}{2})...\Gamma(1)}\\
&= 2^{1/2}a_1 \frac{\Gamma(n+\frac{1}{2})\Gamma(n)\Gamma(n-\frac{1}{2})\Gamma(n-1)...\Gamma(1)} {\Gamma(n+\frac{3}{2})\Gamma(n+1)\Gamma(n+\frac{1}{2})\Gamma(n)...\Gamma(1)}\\
&= 2^{1/2}a_1 \frac{1} {\Gamma(n+1)\Gamma(n+\frac{3}{2})}\\
\end{aligned}
\end{equation*}
which works if I define 
\begin{equation}
A = a_1 2^{1/2}.
\end{equation}
Notice that $A$ is only $a_1$ multiplied by some constant, just like $B$.
Then finally 
\begin{equation}
y(x) = x^{-1/2}[a_0 cos(x) + a_1 sin(x)] = AJ_{1/2}(x) + BJ_{-1/2}(x)
\end{equation}
where 
\begin{equation*}
A = a_1 2^{1/2} , B = a_0\bigg(\frac{\pi}{2}\bigg)^{1/2}.
\end{equation*}
Which is what I wanted to show.
\end{document}